
\section{The Specification}
\label{goto-specification}

We now present  the initial  specification of  the  garbage
collector.  It is  presented as a  transition  system using an informal
notation for  describing a such.   In Appendix~\ref{pvs-specifications}
it is described how we encode transition systems in PVS.

We shall assume a data  structure representing the memory.  The number
of  nodes in the  memory is defined by the  constant {\tt NODES}\@.  The
type {\tt Node} defines the numbers from $0$ to ${\tt NODES} - 1$.  The
constant {\tt  SONS} defines the number  of  cells per node.  The type
{\tt Index} defines the numbers  from $0$ to ${\tt  SONS} - 1$. Hence,
the  memory can  be  thought of a   two-dimensional array, and can  be
declared as in Fig \ref{spec-state}\footnote{The actual PVS specification shown on
page~\pageref{memory-fig} is actually more abstract and does not
specify the memory as being implemented as an array.  We use
an array implementation here for clarity of presentation. }.

\begin{figure}[htb]
\begin{smallsession}
  var M : array[Node,Index] of Node;
\end{smallsession}
\caption{Specification state}
\label{spec-state}
\end{figure}

The memory will be the observed part  of the state ($\Sigma_{o}$ -- see
Definition~\ref{def-refinement-mapping}) throughout all  refinements. 
For example, the node  colouring  structure and other  auxiliary variables
that we later add will be internal.  Recall that an initial segment of
the  nodes are roots, the  number  being defined by  the constant {\tt
  ROOTS}\@.   A number of functions  (e.g., for reading the  state) and
procedures (e.g., for modifying  the  state) are assumed, see Fig \ref{spec-functions}.

\begin{figure}[htb]
\begin{smallsession}
  function  accessible(n:Node):bool;
  function  son(n:Node,i:Index):Node;
  procedure set_son(n:Node,i:Index,k:Node);
  procedure append_to_free(n:Node);
\end{smallsession}
\caption{Functions and procedures used in the specification}
\label{spec-functions}
\end{figure}

The  function {\tt accessible} returns  true if  its argument node is
accessible from one of the roots by  following pointers.  The function
{\tt son} returns  the contents of  cell  {\tt (n,i)}\@.  The  procedure
{\tt set\_son} assigns {\tt k} to the cell identified  by {\tt (n,i)}\@. 
Hence after the  procedure has been called,   this cell now points  to
{\tt k}\@.  The procedure {\tt   append\_to\_free} appends its  argument
node to the list  of free nodes,  assuming that it is  a garbage node. 
The specification consists of  the parallel composition of the mutator
and   the     collector.    The  mutator is shown in Fig. \ref{spec-mutator}.

\begin{figure}[htb]
\begin{smallsession}
  MODIFY : 
    [1] choose n,k:Node; i:Index where accessible(k) -> 
          set_son(n,i,k); 
          goto MODIFY
        end
\end{smallsession}
\caption{Specification of mutator}
\label{spec-mutator}
\end{figure}

A program at  any time during   its execution is  in  one of a  finite
collection of locations  that are identified  by program  labels.  The
above mutator has one such location named  {\tt MODIFY}\@.  Associated
with  each location is  a set of  numbered ({\tt [1]}, {\tt [2]}, \ldots)
rules,    typically of the form   {\tt  p -> s},   where  {\tt p} is a
pre-condition on the  state and  {\tt s}  is an assignment  statement. 
When the program  execution is at this  location, all rules where  the
condition  {\tt p}  is true  in the  current state are  enabled, and a
non-deterministic choice is  made between them,  resulting in the next
state being obtained  by applying the {\tt s}  statement of the chosen
rule to the current state.  The ``{\tt choose x:T where  p(x) -> s end}''
construct represents a set of such rules,  one for each choice of {\tt
  x} within its type {\tt T}\@.  Hence, the mutator repeatedly chooses
two   arbitrary  nodes {\tt  n,k:Node}   and  an arbitrary index  {\tt
  i:Index} such that {\tt  k} is accessible.  The  cell {\tt (n,i)} is
then set to point to {\tt k}\@.  The collector is shown in Fig \ref{spec-collector}.

\begin{figure}[htb]
\begin{smallsession}
  COLLECT :
    [1] choose n:Node where not accessible(n) -> 
          append_to_free(n); 
          goto COLLECT
        end
\end{smallsession}
\caption{Specification of collector}
\label{spec-collector}
\end{figure}

It repeatedly chooses an arbitrary inaccessible node which  is 
then appended  to  the free list   of  nodes. Since   the  node is not
accessible it     is a garbage   node, hence   only garbage  nodes are
collected  (appended),  and this is the   proper  specification of the
garbage collector.  This yields an abstract specification of the behavior
of the collector that is not yet a reasonable implementation.
We need to somehow implement the selection of an inaccessible node.  


\section{The Refinement Steps}
\label{refinement-steps}

In this section we outline how the refinement  is carried out in three
steps,  resulting  in   the  garbage collection   algorithm  described
informally in Section~\ref{informal-specification}\@.  Each refinement is
given  an individual  subsection, which   again is  divided  into a  {\em
  program}  subsection  presenting  the new program,   and a  {\em proof}
subsection  outlining the  refinement  proof.  According  to
Theorem~\ref{theorem-refinement}  a refinement    can be   proved by
identifying a refinement mapping from the concrete  state space to the
abstract state  space,  see Definition~\ref{def-refinement-mapping}\@.  
Hence, each {\em proof} section will consist of a definition of such a
mapping together with a proof that it is a refinement mapping, focusing
on   the  simulation relation required in  item $(3)$ of
Definition~\ref{def-refinement-mapping}\@.  The   PVS encoding  of  the
programs is described in  Appendix~\ref{pvs-specifications}, while the PVS 
encoding of the refinement proofs is described in Appendix~\ref{pvs-proof}\@.


\subsection{First Refinement : Introducing Colours}
\label{first-refinement}

\subsubsection{The Program}

In  the first step, the  collector is refined  to base  its search for
garbage nodes  on a colouring  technique.  The  type {\tt  Colour}  is
defined as {\tt bool},  the set of  Booleans, assumed to represent the
colours {\em black}   (true) and {\em  white} (false)\@.    The global
state must be  extended with  a colouring  of each node  in the memory
(not each cell), and a couple of extra auxiliary variables {\tt Q} and
{\tt L} used for other purposes.  The extended state is shown in Fig. 
\ref{refinement1-state}.

\begin{figure}[htb]
\begin{smallsession}
  var
    M : array[Node,Index] of Node;
    C : array[Node] of Colour;
    Q : Node;
    L : nat;
\end{smallsession}
\caption{First refinement state}
\label{refinement1-state}
\end{figure}

\noindent Three extra operations on this new data structure are needed,
shown in Fig. \ref{refinement1-functions}.

\begin{figure}[htb]
\begin{smallsession}
  procedure set_colour(n:Node,c:Colour);
  function  colour(n:Node):Colour;
  function  blackened():bool;
\end{smallsession}
\caption{Additional functions and procedures used in first refinement}
\label{refinement1-functions}
\end{figure}

The procedure {\tt set\_colour} colours a node either white or black by
updating the variable {\tt C}\@.  
The function {\tt colour} returns the colour of a node. Finally, the function
{\tt blackened} returns true if {\em all accessible nodes are black}\@.
The  mutator is  now  refined  into  the program  which was informally
described   in    Section~\ref{informal-specification}, see Fig. \ref{refinement1-mutator}.

\begin{figure}[htb]
\begin{smallsession}
  MUTATE :
    [1] choose n,k:Nodes; i:Index where accessible(k) -> 
          set_son(n,i,k); 
          Q := k; 
          goto COLOUR;
        end
  COLOUR :  
    [1] true -> set_colour(Q,true); goto MUTATE;
\end{smallsession}
\caption{Refinement of mutator}
\label{refinement1-mutator}
\end{figure}

There are two  locations, {\tt MUTATE} and {\tt  COLOUR}\@.  In the {\tt
  MUTATE} location,  in addition  to  the
mutation, the target node {\tt k} is assigned  to the global auxiliary
variable {\tt  Q}\@.   Then in the  {\tt COLOUR}   location,  {\tt Q} is
coloured black.    {\em Note that    the mutator will not  be  further
  refined, it will now stay unchanged during the remaining refinements
  of the collector}\@. The collector is defined in Fig \ref{refinement1-collector}.

\begin{figure}[htb]
\begin{smallsession}
  COLOUR :
    [1] choose n:Nodes -> 
          set_colour(n,true); 
          goto COLOUR;
        end;
    [2] blackened() -> L := 0; goto TEST_L;
  TEST_L :
    [1] L = NODES -> goto COLOUR;
    [2] L < NODES -> goto APPEND;
  APPEND :
    [1] not colour(L) -> append_to_free(L); L := L + 1; goto TEST_L;
    [2] colour(L) -> set_colour(L,false); L := L + 1; goto TEST_L; 
\end{smallsession}
\caption{First refinement of collector}
\label{refinement1-collector}
\end{figure}

It consists of two phases.  While  in the {\tt COLOUR} location, nodes
are coloured arbitrarily until  all  accessible nodes are black  ({\tt
  blackened()})\@.  The style in which colouring is expressed may seem
surprising, but it is a way of defining a  post condition: {\em colour
  at  least  all   accessible nodes}\@.\footnote{By  formulating  this
  colouring as    an iteration, we   can  avoid introducing  a history
  variable at a lower  refinement level.   Note  that any node can  be
  coloured, not only accessible  nodes. This allows a later refinement
  to colour nodes   that originally were  accessible,  but  later have
  become  garbage.}  In the  second phase at locations {\tt TEST\_L} and
{\tt APPEND},  all white nodes are  regarded as garbage nodes, and are
hence collected (appended to the free list)\@.  The auxiliary variable
{\tt L}  is used to  control the loop: it  runs through all the nodes. 
After  appending all  garbage  nodes to the  free  list, the colouring
phase is restarted.


\subsubsection{The Refinement Proof}
\label{first-informal-refinement}

The  refinement mapping, call it $abs$,  from the concrete state space
to the abstract state space maps {\tt M} to {\tt M}\@.  Note that such
a mapping only needs to be defined  for each component of the abstract
state,  showing how it  is generated  from  components in the concrete
state.  Hence, the concrete variables {\tt C}, {\tt Q} and {\tt L} are
not used  for this  purpose.   This is  generally  the case   for  the
refinement mappings to follow: they are  the identity on the variables
occurring  in the abstract state.  Also  program  locations have to be
mapped.  In fact, each program (mutator, collector) can be regarded as
having a program   counter  variable, and we    have to show  how  the
abstract program   counter is obtained  (mapped)  from  the  concrete. 
Whenever the  concrete program is in a  particular  location $l$, then
the abstract program will be in  the location $abs(l)$.   In the current case,
the concrete mutator locations {\tt  MUTATE} and {\tt COLOUR} are both
mapped to  {\tt MODIFY}, while  the  concrete collector locations {\tt
  COLOUR}, {\tt  TEST\_L}  and {\tt APPEND}   all are mapped   to {\tt
  COLLECT}\@. This completes the definition of the refinement mapping.

In  order      to     prove   Property    $(3)$      in     Definition
\ref{def-refinement-mapping}, we   associate  each transition  in  the
concrete program with a transition  in the abstract program, and prove
that: ``if the  concrete transition brings   a state $s_1$ to  a state
$s_2$, then the abstract transition brings the state $abs(s_1)$ to the
state $abs(s_2)$''.   We say that  the concrete transition, say $t_c$,
{\em simulates} the abstract transition, say $t_a$,  and write this as
$t_c \refines t_a$.  Putting  all these sub-proofs together will yield
a proof of $(3)$.   Some of the  concrete transitions just simulate  a
stuttering step  (no state change) in the  abstract system.  This will
typically be   some of the  new  transitions associated with  {\em new
  location  names}  added  to  the  concrete program.  
Other   concrete transitions have   exact counterparts in the abstract
program.   These are typically transitions  associated  with {\em same
  location names}  as  in the abstract program.  In the
following, we will only mention cases that deviate from the above two;
i.e.,  where  we   add  {\em  new   location   names},  and  where  the
corresponding {\em new} transitions do {\em not} simulate a stuttering
step in the abstract program.

Hence in our case, {\tt MUTATE.1} $\refines$  {\tt MODIFY.1}, and {\tt
  APPEND.1}  $\refines$ {\tt  COLLECT.1}\label{append-refines-collect}
({\tt APPEND.2} simulates   stuttering)\@.    In the proof  of    {\tt
  APPEND.1} $\refines$ {\tt COLLECT.1},  an invariant is needed  about
the concrete program:

\begin{center}
\label{refinement1-inv1}
  {\tt collector@APPEND} $\andthen$ {\tt accessible(L)} 
  $\implies$ {\tt colour(L)}
\end{center}

\noindent
It says that whenever  the concrete collector is  at the {\tt  APPEND}
location, and node {\tt L} is accessible, then  {\tt L} is also black. 
From this we can conclude that the {\tt append\_to\_free} operation is
only applied to garbage nodes, since it is only applied to white nodes.
Hence, we need  to prove an  invariant about  the concrete program  in
order  to  prove  the refinement.  In   general,  the  proof of  these
invariants is what really makes the refinement  proof non-trivial.  To
prove the above invariant, we   do in fact  need  to prove a  stronger
invariant, namely that in locations  {\tt TEST\_L} and {\tt  APPEND}:
$\forall n \ge$ {\tt L} $\cdot$  
{\tt accessible(}$n${\tt )} $\implies$ {\tt colour(}$n${\tt )}. 
This invariant strengthening is typical in our proofs.


\subsection{Second Refinement : Colouring by Propagation}
\label{second-refinement}

\subsubsection{The Program}

In this  step,  accessible nodes  are  coloured  through a propagation
strategy, where first all roots are coloured, and next all white nodes
which have a black father are coloured.  The state is extended with an
extra auxiliary variable {\tt K}  used for controlling the  iteration
through  the  roots.   The   extended    state is shown  in Fig \ref{refinement2-state}.
Two  additional    functions are needed, shown 
in Fig. \ref{refinement2-functions}.

\begin{figure}[htb]
\begin{smallsession}
  var
    M : array[Node,Index] of Node;
    C : array[Node] of Colour;
    Q : Node;
    K, L : nat;
\end{smallsession}
\caption{Second refinement state}
\label{refinement2-state}
\end{figure}

\begin{figure}[htb]
\begin{smallsession}
  function bw(n:Node,i:Index):bool;
  function exists_bw():bool;
\end{smallsession}
\caption{Additional functions used in second refinement}
\label{refinement2-functions}
\end{figure}

The function {\tt bw} returns true if {\tt n} is black and {\tt son(n,i)}
is white. The function {\tt exists\_bw} returns true if there exists a
black node,  say {\tt n},  that  via one of  its cells,  say {\tt i},
points to a white node. That is: {\tt bw(n,i)}\@. The collector becomes as
shown in Fig. \ref{refinement2-collector}.

\begin{figure}[htb]
\begin{smallsession}
  COLOUR_ROOTS :
    [1] K = ROOTS -> goto PROPAGATE;
    [2] K < ROOTS -> set_colour(K,true); K := K+1; goto COLOUR_ROOTS;
  PROPAGATE :
    [1] choose n:Node; i:Index where bw(n,i) -> 
          set_colour(son(n,i),true); 
          goto PROPAGATE;
        end;
    [2] not exists_bw() -> L := 0; goto TEST_L;
  TEST_L :
    [1] L = NODES -> K := 0; goto COLOUR_ROOTS;
    [2] L < NODES -> goto APPEND;
  APPEND :
    [1] not colour(L) -> append_to_free(L); L := L + 1; goto TEST_L;
    [2] colour(L) -> set_colour(L,false); L := L + 1; goto TEST_L; 
\end{smallsession}
\caption{Second refinement of collector}
\label{refinement2-collector}
\end{figure}

The {\tt COLOUR} location from the previous level has been replaced by
the two locations {\tt  COLOUR\_ROOTS} and {\tt PROPAGATE} (while  the
append phase is mostly unchanged)\@. In the {\tt COLOUR\_ROOTS} location
all roots  are  coloured black,  the  loop  being controlled  by   the
variable {\tt K}\@. In the {\tt PROPAGATE} location, either there exists
no black node with a white son (i.e. {\tt not exists\_bw()}), in which
case we start collecting (going to  location {\tt TEST\_L}), or such a
node exists, in which case its son is coloured black, and we continue
colouring.


\subsubsection{The Refinement Proof}

The  refinement mapping,  besides  being  the identity on  identically
named entities (variables  as well as  locations), maps the collector
locations {\tt COLOUR\_ROOTS} and {\tt PROPAGATE} to {\tt COLOUR}\@.  Hence
concrete   root  colouring as well  as   concrete propagation are just
particular kinds of abstract colourings.

Concerning the transitions,  
{\tt  COLOUR\_ROOTS.2} $\refines$ {\tt  COLOUR.1}, 
{\tt PROPAGATE.1} $\refines$ {\tt  COLOUR.1},  
and
{\tt PROPAGATE.2} $\refines$ {\tt COLOUR.2}. 
In   the proof  of {\tt PROPAGATE.2} $\refines$ {\tt COLOUR.2}, 
an invariant is needed about the concrete program:

\begin{center}
  {\tt collector@PROPAGATE} $\implies$ $\forall r:$ {\tt Root} $\cdot$ 
  {\tt colour(}$r${\tt )}
\end{center}

\noindent
It states that in location {\tt PROPAGATE} all roots must be coloured.
This fact combined with the propagation termination condition {\tt not
  exists\_bw()}: ``there does not exist a pointer from a black node to
a  white node'', will  imply  the propagation termination condition in
{\tt  COLOUR.2} of  the  abstract specification:  {\tt blackened()},
which says that ``all accessible nodes are coloured''.


\subsection{Third Refinement : Propagation by Scans}
\label{third-refinement}

\subsubsection{The Program}

In the last refinement,  the propagation, represented by  the location
{\tt PROPAGATE}  above, is refined into an  algorithm, where all nodes
are repeatedly  scanned in sequential  order, and if black, their sons
coloured;  until a  whole scan  does not  result in  a colouring.  The
state   is  extended with auxiliary  variables   {\tt  BC} ({\em black
  count}) and {\tt OBC}   ({\em old black  count}), used  for counting
black  nodes; and the  variables {\tt  H}, {\tt I},   and {\tt J}  for
controlling loops, see Fig. \ref{refinement3-state}.

\begin{figure}[htb]
\begin{smallsession}
  var
    M   : array[Node,Index] of Node;
    C   : array[Node] of Colour;
    Q   : Node;
    H, I, J, K, BC, OBC : nat;
\end{smallsession}
\caption{Third refinement state}
\label{refinement3-state}
\end{figure}

\begin{figure}[htb]
\begin{smallsession}
- Step 1 : Colour roots
  COLOUR_ROOTS : 
    [1] K = ROOTS -> I := 0; goto TEST_I;
    [2] K < ROOTS -> set_colour(K,true); K := K + 1; goto COLOUR_ROOTS;
- Step 2 : Propagate once
  TEST_I :
    [1] I = NODES -> BC := 0; H := 0; goto TEST_H;
    [2] I < NODES -> goto TEST_COLOUR;
  TEST_COLOUR :
    [1] not colour(I) -> I := I + 1; goto TEST_I;
    [2] colour(I)  -> J := 0; goto COLOUR_SONS;
  COLOUR_SONS :
    [1] J = SONS -> I := I + 1; goto TEST_I;
    [2] J < SONS -> set_colour(son(I,J),true); J := J + 1; 
        goto COLOUR_SONS;
- Step 3 : Count black nodes
  TEST_H :
    [1] H = NODES -> goto COMPARE;
    [2] H < NODES -> goto COUNT;
  COUNT :
    [1] not colour(H) -> H := H + 1; goto TEST_H;
    [2] colour(H) -> BC := BC + 1; H := H + 1; goto TEST_H;
  COMPARE :
    [1] BC = OBC -> L := 0; goto TEST_L;
    [2] BC /= OBC -> OBC := BD; I := 0; goto TEST_I;
- Step 4 : Append garbage nodes
  TEST_L :
    [1] L = NODES -> BC := 0; OBC := 0; K := 0; goto TEST_I;
    [2] L < NODES -> goto APPEND;
  APPEND :
    [1] not colour(L) -> append_to_free(L); L := L + 1; goto TEST_L;
    [2] colour(L) -> set_colour(L,false); L := L + 1; goto TEST_L; 
\end{smallsession}
\caption{Third and final refinement of collector}
\label{refinement3-collector}
\end{figure}

\noindent The collector is described in Fig \ref{refinement3-collector}, 
where transitions have been 
divided into 4 steps corresponding  to the informal description of the
algorithm    on page    \pageref{the-collector-informal}\@.   Two  loops
interact (steps 2   and 3)\@.  In the   first loop, {\tt  TEST\_I}, {\tt
  TEST\_COLOUR} and   {\tt COLOUR\_SONS}, all   nodes are scanned, and
every black node has all its sons coloured. The variables {\tt I} and
{\tt J} are  used to ``walk'' through the  cells.  In the second loop,
{\tt TEST\_H}, {\tt COUNT}  and {\tt COMPARE},  it is counted how many
nodes are black.  This amount is stored in the  variable {\tt BC}, and
if this amount exceeds  the old  black count,  stored in  the variable
{\tt OBC}, then yet another scan is started, and {\tt OBC} is updated.
The variable {\tt H} is used to control this loop.


\subsubsection{The Refinement Proof}

The refinement mapping  is the identity, except  for six of the locations
of  the collector.  That is,  the collector locations {\tt TEST\_I}, {\tt
  TEST\_COLOUR}, {\tt COLOUR\_SONS}, {\tt  TEST\_H}, {\tt COUNT},  and
{\tt COMPARE} are  all mapped to  {\tt  PROPAGATE}\@. 
Concerning   the  transitions,  
{\tt  COLOUR\_SONS.2} $\refines$ {\tt PROPAGATE.1} 
whereas  
{\tt COMPARE.1} $\refines$ {\tt PROPAGATE.2}.    
In  the   proof   of  
{\tt  COLOUR\_SONS.2} $\refines$ {\tt PROPAGATE.1}, 
the following invariant is needed:

\begin{center}
  {\tt collector@COLOUR\_SONS} $\implies$ {\tt colour(I)}
\end{center}

\noindent
This property    implies   that the  abstract   {\tt    PROPAGATE.1}
transition pre-condition {\tt bw(I,J)} will be true (in case the son
is white)  or otherwise  (if  the  son is  also  black),  the concrete
transition corresponds to  a  stuttering  step (colouring an   already
black son is the identity function)\@.  Correspondingly, in the proof of
{\tt  COMPARE.1}  $\refines$  {\tt   PROPAGATE.2},  the  following
invariant is needed:

\begin{center}
  {\tt collector@COMPARE} $\andthen$ {\tt BC = OBC} $\implies$ 
  $\neg$ {\tt exists\_bw()}
\end{center}

\noindent
It states that when the collector is in  location {\tt COMPARE}, after
a counting scan where the number of black  nodes have been counted and
stored in {\tt  BC}, if the number counted  equals the  previous (old)
count {\tt OBC} then there does not exist a pointer  from a black node
to a   white  node.  Note that   {\tt  BC =   OBC} is  the propagation
termination condition,  and this  then corresponds to  the termination
condition {\tt  not   exists\_bw()} of the  abstract  transition  {\tt
  PROPAGATE.2}.  The  proof  of    these  two invariants   is  quite
elaborate,   and does in  fact  compare  in  size  and ``look'' to the
complete  proofs in \cite{havelund-pvs-gc-99}  as well as in \cite{Rus:GC}\@. 

%%% Local Variables: 
%%% mode: latex
%%% TeX-master: t
%%% End: 
