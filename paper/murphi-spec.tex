         
\chapter{Model Checking in Murphi}

In this chapter, we shortly outline our experience with encoding the
garbage collector in the Murphi model checker \cite{MDC:Murphi}.
The full  formal Murphi program is contained
in appendix \ref{app:murphi-spec}.

Murphi uses a  program  model  that is  similar   to those  of   UNITY
\cite{CM:UNITY} and TLA \cite{TLA:TOPLAS94}, hence the  one we have used in
our PVS   specification.  A Murphi  program  has three   components: a
declaration  of the  global variables,  a description  of the  initial
state, and a collection of  transition rules.  Each transition rule is
a guarded command that consists of a boolean guard expression over the
global  variables,  and  a deterministic  statement   that changes the
global variables.  In addition, one  can state invariant conditions to
be verified.

An execution of a Murphi program is obtained by repeatedly {\it (1)
arbitrarily selecting  one of the transition  rules  where the boolean
guard is true in the current state; (2) executing the statement of the
chosen transition  rule}\@.  The statement  is executed atomically: no
other  transition  rules  are   executed   in  parallel.   Thus  state
transitions  are interleaving  and processes  communicate  via  shared
variables.  The notion of process is not  formally  supported, but may
be thought  of  as  a  subset  of  the transition  rules.   The Murphi
verifier tries to explore all reachable states in order to ensure that
all  invariants hold.  If a violation is detected,  Murphi generates a
violating trace.

The reader  is referred to the  appendix for the details of the model.
Here we shall  focus on the differences between the PVS model  and the
Murphi model. The two principal advantages of PVS are that (1): in PVS
we can  verify  a  parameterized program,  whereas  in  Murphi we  are
limited to a finite state program; and: (2) in PVS we can be  abstract
at  the algorithmic level,  whereas  in Murphi we have to make certain
design choices.  The  advantage  of  Murphi  is that  verification  is
automatic, whereas in PVS we have to manually assist the proof.


\section*{Infinite Versus Finite State}

The first obvious difference concerns the size of the memory.   In PVS,
the   boundaries  ({\tt  NODES},  {\tt  SONS}  and  {\tt  ROOTS})  are
unspecified    parameters   (figure   \ref{memory-fig}),   hence   the
correctness does not depend  on  their specific  values.  The  garbage
collector is therefore verified for any size of memory.  In the Murphi
program,  on  the  other hand, we  have to fix the boundaries to
particular natural numbers.  In  our case, we  verified  the algorithm
with the  following  values: {\tt NODES} = 3, {\tt SONS} =  2 and {\tt
ROOTS = 1} (figure \ref{murphi-bounds-fig}).

\begin{figure}
\begin{smallsession}
Const
  NODES : 3;
  SONS  : 2;
  ROOTS : 1;
\end{smallsession}
\caption{The Choice of Fixed boundaries}
\label{murphi-bounds-fig}
\end{figure}

In  this context, Murphi used  2895  seconds to verify  the invariant,
exploring 415633 states and firing 3659911 transition rules. It turned
out that Murphi was unable to verify bigger memories within reasonable
time (days).


\section*{Abstractness of Memory}

The PVS memory is abstractly specified in terms of a set of axioms
(figure  \ref{memory-fig}).
Although we think of the memory as an array of two dimensions, this is in
fact not required by an implementation. In the Murphi program, on the
other hand, we need to choose an implementation of the memory, and we have
chosen to model it as a two dimensional array as illustrated in figure
\ref{murphi-memory-fig}.

\begin{figure}
\begin{smallsession}
Type
  Node   : 0..NODES-1;
  Index  : 0..SONS-1;
  Colour : boolean;
  NodeStruct : Record
                 colour : Colour;
                 cells  : Array[Index] Of Node;
               End;

Var
  M : Array[Node] Of NodeStruct; 

Function colour(n:Node):Colour;
Begin
  Return M[n].colour;
End;

Procedure set_colour(n:Node;c:Colour);
Begin
  M[n].colour := c;
End;

Function son(n:Node;i:Index):Node;
Begin
  Return M[n].cells[i]
End;

Procedure set_son(n:Node;i:Index;k:Node);
Begin
  M[n].cells[i] := k;  
End;
\end{smallsession}
\caption{The Memory}
\label{murphi-memory-fig}
\end{figure}


\section*{Abstractness of the Append Operation}

The  PVS append operation is abstractly specified in terms of a set of
axioms (figure \ref{append-fig}). Hence  we have made no decisions for
example as to where is the head of the free list, or whether to append
elements first  or  last.   In Murphi  we are  obliged  to  take  such
decisions.  Figure \ref{murphi-append-fig}  shows how we  have  chosen
cell (0,0) to be the head of  the list,  such  that  new elements  are
added to the front.

\begin{figure}
\begin{smallsession}
Procedure append_to_free(new_free:Node);
Var
  old_first_free : Node;
Begin
  old_first_free := son(0,0);
  set_son(0,0,new_free);
  For i:Index Do set_son(new_free,i,old_first_free) EndFor;
End;
\end{smallsession}
\caption{The {\tt append\_to\_free} Operation}
\label{murphi-append-fig}
\end{figure}


\section*{Abstractness of the Accessibility Predicate}

The PVS {\tt accessible} predicate is specified in an abstract manner
using     existential    quantification     over     paths     (figure
\ref{accessible-fig}).  Such a  formulation is not possible in Murphi,
where we  have to  code an algorithm  that marks nodes already visited
during the  examination of  accessible  nodes.  This  is  to  avoid  a
looping  behaviour in case  of  cyclic  accessibility  in  the  memory.
Figure \ref{murphi-accessible-fig} illustrates the algorithm.

\begin{figure}
\begin{smallsession}
Function accessible(n:Node):boolean;
Type
  Status : Enum\{TRY,UNTRIED,TRIED\};
Var
  status : Array[Node] Of Status;
  s : Node;
  try_again : boolean;
Begin
  For k:Node Do
    status[k] := (is_root(k) ? TRY : UNTRIED)
  EndFor;
  try_again := true;
  While try_again Do
    try_again := false;
    For k:Node Do
      If status[k]=TRY Then
        For j:Index Do
          s := son(k,j);
          If status[s]=UNTRIED Then 
            status[s] := TRY; 
            try_again := true;            
          End;
        EndFor;
        status[k] := TRIED;
      End;
    EndFor;
  End; 
  Return status[n]=TRIED
End;
\end{smallsession}
\caption{The {\tt accessible} Predicate}
\label{murphi-accessible-fig}
\end{figure}


%%% Local Variables: 
%%% mode: latex
%%% TeX-master: t
%%% End: 
