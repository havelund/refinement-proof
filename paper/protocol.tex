

\chap{The Bounded Retransmission Protocol}{protocol}

\renewcommand{\mth}[1]{$#1$}

The bounded retransmission protocol developed at Philips Research
Laboratory communicates messages  from a  {\em  producer} to a  {\em
consumer}  over an  unreliable physical  medium that can lose 
messages.  The protocol  is a
nontrivial extension  of the  alternating bit
protocol~\cite{Bartlett69:ABP} that uses timeouts and aborts transmission
following a bounded number of retransmission attempts.   
The {\em environment}  of the protocol consists of
the producer  and the consumer.  The black box view of the system
is that it accepts requests REQ(\mth{f}) from the producer to transmit
the file \mth{f}\@.  When transmission of a file
has been either completed or aborted, the producer receives a
confirmation CONF(\mth{c}), where \mth{c} is either OK, NOT\_OK, or
DONT\_KNOW, respectively indicating that the file was successfully
transmitted, aborted, or that the last message in the file was not
acknowledged but might have been received by the consumer.
The consumer either receives an IND\_ERR signal indicating that
the file transmission was aborted, or an IND(\mth{m}, \mth{i}) signal where
\mth{m} is the message and \mth{i} is either FIRST, LAST,
or
INCOMPLETE
corresponding to the first, last, or an intermediate message in the file.
\comment{
The system inputs a file of messages from
the producer and transmits these to consumer, one message at a time.
grouped in files, and files  arrive from the producer, one at
a time,  and  are then  send  via  the  unreliable  medium, message by
message,  until   the whole file  has been   transmitted, or until the
transmission is cancelled due to loss in the unreliable medium.
}

The protocol consists of a {\em sender}  program at the producer side;
a {\em  receiver} program  at the  consumer side,  and two channels: a
message channel K, and an acknowledgment channel
L.  Both  channels are unreliable in that they can lose
messages or acknowledgments.    The
protocol is  pictured in  Figure~\ref{protocol-figure}.  The   sender
sends each message  over the  channel K  and then waits  for an
acknowledgment    on    channel L.   If   there is no
acknowledgment, the sender times  out and retransmits the
message.   There  is   a fixed  upper   bound on    the  number   of   such
retransmissions.   Communication is asynchronous: K and
L are actually unbounded buffers but since there is at most
one message active in a channel at any point in time, these are modeled here
for simplicity as one-place buffers.  This simplification
is also present in the work of Helmink, {\em et al\/}~\cite{HSV:Protocol.Coq}.

\begin{figure}[htb]
\makebox[\textwidth]{\epsfbox{protocol.figure.ps}}
\caption{The BRP Protocol}
\label{protocol-figure}
\end{figure}

The protocol uses three timers to deal with loss of messages and
acknowledgments.  A timer has a fixed period $T$ of time associated.  When
it is {\em set}, a {\em timeout} occurs $T$ time units or more later. The
first timer in the sender is used to detect the loss of a message or an
acknowledgment.  It is used as follows: when the sender sends a new
message over K, timer~1 is set.  The time associated with this
timer exceeds the time it takes from when a message has been sent over
K until the corresponding acknowledgment is received over L.
If an acknowledgment comes back within this time, the timer is {\em
cleared} (and the next message is sent).  If not, a timeout occurs
whereupon the message is retransmitted, and the timer is set again.  When
the retransmission bound has been reached, the sender aborts transmission
and confirms that the transmission failed.  Either it confirms
CONF(NOT\_OK) or it confirms CONF(DONT\_KNOW)\@.  Two other
timers are used to bring the sender and the receiver back in synchrony
after a file transmission has been aborted by the sender.  We do not model
the real-time aspects of the protocol but instead represent the timers by
timer events.  For example, the timeout event which is supposed to detect
message loss is instead defined to occur when a message is lost.  This
simplification is also present in Helmink, {\em et
al\/}~\cite{HSV:Protocol.Coq}\@.

The receiver may also retransmit  acknowledgments.  This  happens  when
the  receiver gets a 
message that it has already received once.  The receiver distinguishes
between an old message and a new  message via the alternating bit (the
toggle) which is part of the message.

We can now examine the behavior of the protocol for the case when no
messages or acknowledgments are lost.  The sender sends each individual
message \mth{m} in the file to the receiver over the channel K in the form
(first,last,toggle,\mth{m})\@.  If the producer signals
REQ($f$) where $f$ is $[m_1, m_2, m_3]$, 
then the sender first sends the tuple (true,false,toggle,$m_1$)
on channel K\@.  Upon receipt of this message, the receiver signals
IND($m_1$,FIRST) to the consumer and sends an acknowledgment on channel
L\@.  The sender then sends the second message as 
(false,false,$\neg$ toggle,$m_2$).  The receiver correspondingly
signals IND($m_2$,INCOMPLETE) and acknowledges receipt.
The sender then sends the last message as 
(false,true,toggle,$m_3$)\@.  The receiver now 
signals IND($m_3$,LAST) and acknowledges receipt.  The sender
on receiving this acknowledgment signals CONF(OK) to confirm
successful transmission to the producer.

One complication in the protocol is that  upon abortion of a file
transmission, the sender and the receiver must regain synchrony for the
next file transmission.  If the retransmission bound for a message has
been reached without an acknowledgment, then the file transmission is
aborted.  The sender must then delay for a sufficiently long interval
before transmitting the next file so that the
receiver's timer can time out and the receiver can hence determine 
that file transmission has been aborted.  Another complication is that
when file transmission is aborted, then it is possible that the next
message on channel $K$ (namely, the first message of a new file) may not
have its alternating bit properly toggled.  This is because the receiver
may have received the earlier message but the sender did not receive an
acknowledgment.  To cope with this loss of synchronization, the receiver
resets its toggle bit to that of the first message and uses the bit
alternation for subsequent messages in the same file.

It is not evident  how to formulate the  correctness criteria for this
protocol.  First of all, we limit ourselves to safety properties (that
some property always  holds),   since already these  seem  challenging
enough  when trying to do  ``real'' proofs.  Liveness properties (that
some property  eventually holds)  are of  course of big  interest, but
they are generally   harder  to prove.   Safety  properties  are  also
sometimes called invariants.   But   even  after  having    restricted
ourselves to safety properties, the property to prove is not evident.

One  solution  is to state  some invariant  over the  variables of the
program.  Suppose  that the variables are $x_1$,\ldots,$x_n$,  then
we could establish some  predicate $P$  over these variables, and
then show  that $P(x_1,\ldots,x_n)$ always  holds. It is, however,
difficult  to come up  with  a predicate  $P$  that  gives a good
intuitive feeling of correctness.

The solution we shall  choose mimics the correctness criteria  defined
in \cite{HSV:Protocol.Coq},   where  the protocol  is  proved  to be a
refinement of   a  more  abstract   protocol.  We  shall   not prove a
refinement, but we shall formulate  the correctness criteria in  terms
of a more  abstract protocol, the  same as  in \cite{HSV:Protocol.Coq}
basically.  Instead of proving a  refinement, we let  the
abstract protocol ``run in  parallel'' with the  concrete one: for every
occurrence of an external communication (REQ, IND,
CONF, or IND\_ERR) in the concrete protocol, the abstract 
protocol analyses whether this move  is allowed, and  a flag is raised
in case  the ``laws'' of the abstract  protocol are broken.

The   abstract protocol    only  specifies how   the events  REQ($f$),
IND($m$,$i$), CONF($c$) and IND\_ERR  can occur, ignoring  the details
of the   channels    K and L. That     is,  it only  focuses    on the
communications  with  the  environment   according  to the  black  box
view.  We  shall  see below  how   this is  formulated.  The  parallel
composition of  the concrete and  the  abstract protocols has the  big
advantage  that it can  be model checked in the  finite-state case.
