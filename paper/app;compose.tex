

\newpage
\subsection{Final Refinement Theorem}

\begin{alltt}
%%%%%%%%%%%%%%%%%%%%%%%%%%%%%%%%%%%%%%%%%%%%%%%%%%%%%%%%%%%%%%%%%%%%
% Composed_Refinement :                                            %
%   States the final correctness criteria in theorem ``ref'',      %
%   which says that the third implementation refines the top-level %
%   specification. The proof uses the transitivity property of the %
%   refinement relation.                                           %
%%%%%%%%%%%%%%%%%%%%%%%%%%%%%%%%%%%%%%%%%%%%%%%%%%%%%%%%%%%%%%%%%%%%
 
Composed_Refinement[
  NODES : posnat,
  SONS  : posnat,
  ROOTS : posnat] : THEORY
 
BEGIN
 
  ASSUMING
    roots_within : ASSUMPTION ROOTS <= NODES
  ENDASSUMING

  IMPORTING Refinement1[NODES,SONS,ROOTS]
  IMPORTING Refinement2[NODES,SONS,ROOTS]
  IMPORTING Refinement3[NODES,SONS,ROOTS]

  IMPORTING Refine_Predicate
  IMPORTING Refine_Predicate_Transitive

  ref2 : LEMMA 
    refines[S.O_State,S.State,I2.State]
      (I2.init,I2.next,I2.proj)(S.init,S.next,S.proj)

  ref : THEOREM 
    refines[S.O_State,S.State,I3.State]
      (I3.init,I3.next,I3.proj)(S.init,S.next,S.proj)

END Composed_Refinement
\end{alltt}

%%% Local Variables: 
%%% mode: latex
%%% TeX-master: t
%%% End: 
