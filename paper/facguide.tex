% FACGUIDE.TEX
% Copyright (C) 1991 Cambridge University Press
% Version 1.00, released 27 September 1991

% The following macros define symbols to be used in Table 1 of
% the authors' guide, using characters from AMS founts.
% If you have access to AMS founts, uncomment the
% first three paragraphs at the beginning of this file, i.e.
% "\font\fivemsx..." to "...\def\whwh{\white\kern-1pt--\white}"
% and the first 4 entries of the table.

\documentstyle{fac}

% \font\fivemsx=msxm10 at 5pt
% \newfam\msxfam
% \textfont\msxfam=\fivemsx
% \def\hexnumber@#1{\ifcase#1 0\or1\or2\or3\or4\or5\or6\or7\or8\or9\or
% 	A\or B\or C\or D\or E\or F\fi }
% \edef\msx@{\hexnumber@\msxfam}

% \mathchardef\blacktriangleright="3\msx@49
% \mathchardef\vartriangleright="3\msx@42

% \def\black{\raisebox{.2ex}{$\blacktriangleright$}}
% \def\white{\raisebox{.2ex}{$\vartriangleright$}}
% \def\whbl{\white\kern-1pt--\black}
% \def\blwh{\black\kern-1pt--\white}
% \def\blbl{\black\kern-1pt--\black}
% \def\whwh{\white\kern-1pt--\white}

\def\eg{{\it e.g.\ }}
\def\etc{{\it etc}}
\def\visiblespace{\leavevmode\hbox{\tt\char`\ }}
\def\LaTeX{L\kern-.36em\raise.3ex\hbox{a}\kern-.15em
    T\kern-.1667em\lower.7ex\hbox{E}\kern-.125emX}

\newtheorem{theorem}{Theorem}[section]

\title[Formal Aspects of Computing: \LaTeX\ Submissions]
      {Formal Aspects of Computing:\\
       \LaTeX\ Style Guide for Authors}

\author[C. Notarmarco and R. Mulvey]
    {Christiane Notarmarco$^1$ and Rod Mulvey$^2$\\
      $^1$Springer-Verlag London Limited, Springer House,
      Wimbledon, SW19 7JZ, UK; $^2$\TeX-to-type,\\
      Cambridge University Press, Shaftesbury Road,
      Cambridge, CB2 2BS, UK}

\correspond{Christiane Notarmarco, Springer-Verlag London
            Limited, Springer House, 8 Alexandra Road,
            Wimbledon, London, SW19 7JZ, UK.}

\date{September\/ \em 1991}
\pagerange{\pageref{firstpage}--\pageref{lastpage}}

\begin{document}
\label{firstpage}

\makecorrespond
\maketitle

\begin{keywords}
\LaTeX; Style files; \verb"fac.sty"; Sample text; User guide
\end{keywords}

\begin{abstract}
This guide is for authors who are preparing papers for
{\em Formal Aspects of Computing\/} using the \LaTeX\ document
preparation system and the FAC style file.
\end{abstract}

\section{Introduction}

In addition to the standard submission of hard-copy from authors,
{\em Formal Aspects of Computing\/} accepts machine-readable forms of
papers in \LaTeX.
The layout design for {\em Formal Aspects of Computing\/} has been
implemented as a \LaTeX\ style file.
The FAC style is based on the \verb"ARTICLE" style as discussed in the
\LaTeX\ manual \cite{Lam:LaTeX}.
Commands which differ from the standard \LaTeX\ interface, or which are
provided in addition to the standard interface, are explained in this
guide. This guide is not a substitute for the \LaTeX\ manual itself.
Authors planning to submit their papers in \LaTeX\ are advised to use
\verb"fac.sty" as early as possible in the creation of their files.

\subsection{Introduction to \LaTeX}

\LaTeX\ is constructed as a series of macros on top of the \TeX\
typesetting program.
\LaTeX\ adds to \TeX\ a collection of facilities which simplify typesetting
for authors by allowing them to concentrate on the logical structure of the
document rather than its visual layout.
Careful use of the \LaTeX\ mark-up philosophy results in uniform
layout rather than the {ad hoc} results of some word-processing systems.
Authors are advised to let the defaults control fount selection, \etc.\
rather than tinker themselves.

\LaTeX\ provides a consistent and comprehensive document preparation
interface. There are macros for generating a table of contents,
lists of figures and/or tables;
\LaTeX\ can automatically number list entries, equations, figures, tables,
and footnotes, as well as parts, sections and subsections.
Using this numbering system, bibliographic citations, page references and
cross references to any other numbered entity (\eg sections, equations,
figures, list entries) are straightforward.

\subsection{The FAC Document Style}

The use of document styles allows a simple change of style (or style
option) to transform the appearance of your document.
The Springer-Verlag London Limited (SVL) FAC style file preserves the
standard \LaTeX\ interface such that any document which can be produced
using the standard \LaTeX\ \verb"ARTICLE" style, can also be produced with
the FAC style.
However, the measure (or width of text) is wider than the default for
\verb"ARTICLE" therefore line breaks will change and long equations may
need re-setting.
Authors are urged to use \verb"fac.sty" from the beginning of their
document preparation; in particular, they are advised not to use wider
measures as given by \verb"a4.sty", \etc.\ because this will involve
changes after they think their submissions are technically accurate.

\subsection{General Style Issues}

Use of \LaTeX\ defaults will result in a pleasing uniformity of layout
and fount selection. Authors should resist the temptation to make
{\em ad hoc\/} changes to these.
Whether using the default founts or not, displayed mathematics should
be set in a distinct fount from running text and the same fount should
be used where formulae occur in running text.
In addition to the standard \TeX\ and \LaTeX\ founts, authors can use the
AMS {\em msxm\/} and {\em msym\/} founts.  The AMS {\em msym\/}
founts have been used in Table~\ref{symbols} to define some of the characters in the first four lines.

The final makeup will use Times Roman founts and if these are
available to authors (\eg via PSLaTeX) they should be employed in order
to ensure good use of space (they are in general more economical than
computer modern founts).

For general style issues, authors are referred to the `Instructions for
authors' in the FAC journal. Authors who are interested in the detail of style
are referred to \cite{But:Copy} and \cite{Chicago}. The language of the
journal is British English and spelling should conform to this.

Use should be made of symbolic references (\verb"\ref") in order to
protect against late changes of order, \etc.

\subsection{Submission of \LaTeX\ Articles to the Journal}

Authors who intend to submit a \LaTeX\ article to FAC should obtain a
copy of the journal's style file \verb"fac.sty". This is available on
request in the first instance from  Springer-Verlag London Limited; remember to
specify the type of media you require. Alternatively, you can download
the file from the Aston \TeX\ archive at \verb"uk.ac.tex" on JANET. The files
\verb"fac.sty" and \verb"facguide.tex" will be found in the directory
\verb"[TEX-ARCHIVE.LATEX.CONTRIB.PUBLICATIONS]". If you
cannot obtain the journal style file, use \verb"ARTICLE" style.

When submitting the final article, ensure that  the following are included:
\begin{enumerate}
 \item Hardcopy printout of the article;
 \item The input file;
 \item A copy of all user-defined macros;
 \item Bibliography files, or if you have used BIB\TeX, the \verb".bib"
                 and \verb".bbl" files;
 \item Any other files necessary to prepare the article
           for typesetting.
\end{enumerate}
The files for the {\em final\/} article should be text-only with no
system dependent control codes. Submit this, if at all possible, by
e-mail over JANET, and the hardcopy by post to D.\,J. Cooke (see the
inside-cover of the journal for correct e-mail and postal addresses).
If you do not have access to e-mail, send the files on disc -- PC $5
{1\over4\,}$in.\ or Apple $3 {1\over 2\,}$in.\ -- along with the hard copy to D.\,J. Cooke.

\section{Using the FAC Style}

First, copy the file \verb"fac.sty" into an appropriate subdirectory on
your system.
The FAC document style is implemented as a complete document style {\em
not\/} a document style option.
In order to use the FAC style, replace \verb"article" by \verb"fac" in the
\verb"\documentstyle" command at the beginning of your document:
%
\begin{verbatim}
  \documentstyle{article}
\end{verbatim}
is replaced by,
\begin{verbatim}
  \documentstyle{fac}
\end{verbatim}
%
In general, the following standard document style options should {\em
not\/} be used with the FAC style:
%
\begin{itemize}
  \item {\tt 10pt}, {\tt 11pt}, {\tt 12pt} -- unavailable.
  \item {\tt draft}, {\tt twoside} (no associated style file) -- {\tt
     twoside} is the default.
  \item {\tt fleqn}, {\tt leqno}, {\tt titlepage}, {\tt twocolumn} --
        should not be used (\verb"fleqn" is already incorporated into
        the FAC style).
\end{itemize}
%
However, {\tt proc}, {\tt ifthen}, {\tt bezier} -- can be used if
necessary.

\section{Additional Facilities}

In addition to all the standard \LaTeX\ design elements, the FAC style
includes the following features:
%
\begin{itemize}
  \item Extended commands for specifying a short version of the title and
        author(s) for the running headlines.
  \item \verb"\correspond" and \verb"\makecorrespond" commands for printing
     the name and address for further correspondence.
  \item A \verb"keywords" environment.
  \item A \verb"proof" environment.
  \item Control of enumerated lists.
\end{itemize}
%
In general, once you have used the additional \verb"fac.sty"
facilities in your document, do not process it with a standard \LaTeX\
style file.

\subsection{Titles and Author's Name}

In the FAC style, the title of the article and the author's name (or
authors' names) are used both at the beginning of the article for the main
title and throughout the article as running headlines at the top of every
page.
The title is used on odd-numbered pages (rectos) and the author's name
appears on even-numbered pages (versos).
Although the main heading can run to several lines of text, the running
head line must be a single line.
Moreover, the main heading can also incorporate new line commands
(\eg \verb"\\") but these are not acceptable in a running headline.
To enable you to specify an alternative short title, which should not be
more than 48 characters and spaces, and an alternative short author's name,
the standard \verb"\title" and \verb"\author" commands have been extended
to take an optional argument to be used as the running headline:
%
\begin{verbatim}
 \title[Formal Aspects of Computing: \LaTeX\ submissions]
        {Formal Aspects of Computing:\\
        \LaTeX\ Style Guide for Authors}
\end{verbatim}
and,
\begin{verbatim}
  \author[C. Notarmarco and R. Mulvey]
    {Christiane Notarmarco$^1$ and Rod Mulvey$^2$\\
      $^1$Springer-Verlag London Limited, Springer House,
      Wimbledon, SW19 7JZ, UK; $^2$\TeX-to-type,\\
      Cambridge University Press, Shaftesbury Road,
      Cambridge, UK}
\end{verbatim}

You may wish to add a \verb"\thanks" note, which produces a footnote to the
title or author. In the latter case, the superscripts by the authors' names
will be automatically generated.

\subsection{Correspondence Note}

The footnote at the bottom of the first page is generated automatically,
and begins with the words {\it Correspondence and offprint
requests to\/{\rm:\visiblespace }}.
The appropriate name and address must be given as an argument to the macro
\verb"\correspond" in the preamble:
%
\begin{verbatim}
  \correspond{Christiane Notarmarco, Springer-Verlag
              London Limited, Springer House, 8 Alexandra
              Road, Wimbledon, London, SW19 7JZ, UK.}
\end{verbatim}
%
This information is subsequently used by the command
\verb"\makecorrespond". This may be placed anywhere
after \verb"\begin{document}"  within the text of the first page. If you
have other footnotes on the first page, \verb"\makecorrespond" should
follow them so as to make {\em Correspondence and \ldots} print at the foot
of the page.

If you omit the \verb"\correspond" and \verb"\makecorrespond" commands,
the name in the correspondence note defaults to the short author's name.

\subsection{Keywords and Abstracts}

At the beginning of your article, the title should be generated in the
usual way using the \verb"\maketitle" command. Immediately following the
title you should include a list of keywords followed by an abstract. For
example, the titles for this guide were produced by the following source:
%
\begin{verbatim}
  \maketitle
  \begin{keywords}
    \LaTeX; Style files; \verb"fac.sty"; Sample text;
    User guide
  \end{keywords}

  \begin{abstract}
    This guide is for authors who are preparing papers
    for {\em Formal Aspects of Computing\/} using the
    \LaTeX\ document preparation system and the FAC style
    file.
  \end{abstract}

  \section{Introduction}
  ...
\end{verbatim}
The abstract is automatically followed by a 12~pica rule.

\subsection{Proofs}

The \verb"proof" environment has been added to the standard \LaTeX\
constructs to provide a consistent format for proofs.
For example,
%
\begin{verbatim}
  \newtheorem{theorem}{Theorem}[section]
  ...
  \begin{theorem}
    $VS \Rightarrow FP$ for program {\em Multi-States}.
  \end{theorem}
  \begin{proof}
    Noting that a state satisfying $VS$ has no elements
    $V_{xi}$ of $V$ satisfying $0<V_{xi}<1$, it follows
    that the proof is just as the proof was for the
    corresponding case of {\em Bi-States}.
  \end{proof}
\end{verbatim}
%
produces the following text:
%
  \begin{theorem}
    $VS \Rightarrow FP$ for program {\em Multi-States}.
  \end{theorem}
  \begin{proof}
    Noting that a state satisfying $VS$ has no elements
    $V_{xi}$ of $V$ satisfying $0<V_{xi}<1$, it follows
    that the proof is just as the proof was for the
    corresponding case of {\em Bi-States}.
  \end{proof}

\subsection{Lists}

The FAC style provides the three standard list environments:
\begin{itemize}
  \item Numbered lists, created using the \verb"enumerate" environment.
  \item Bulleted lists, created using the \verb"itemize" environment.
  \item Labelled lists, created using the \verb"description" environment.
\end{itemize}
The enumerated list numbers each list item with an arabic numeral.
Alternative styles can be achieved by inserting a redefinition of the
number labelling command after the \verb"\begin{enumerate}".
For example, a list numbered with roman numerals inside parentheses can be
produced by the following commands:
%
\begin{verbatim}
  \begin{enumerate}
   \renewcommand{\theenumi}{(\roman{enumi})}
   \item first item
         :
  \end{enumerate}
\end{verbatim}
%
This produces the following list:
%
\begin{enumerate}
  \renewcommand{\theenumi}{(\roman{enumi})}
  \item first item
  \item second item
  \item \etc\ldots
\end{enumerate}
%
In the last example, the labels were pushed out into the margin  because
the standard list indentation is designed to be sufficient for arabic
numerals rather than the longer roman numerals. In order to enable
different labels to be used more easily, the \verb"enumerate" environment
in the FAC style can be given an optional argument which (like a standard
\verb"bibliography" environment) specifies the {\em widest label}. For
example,
%
\begin{enumerate}[(iii)]
\renewcommand{\theenumi}{(\roman{enumi})}
  \item first item
  \item second item
  \item \etc\ldots
\end{enumerate}
%
was produced by the following input:
%
\begin{verbatim}
  \begin{enumerate}[(iii)]
   \renewcommand{\theenumi}{(\roman{enumi})}
   \item first item
         :
  \end{enumerate}
\end{verbatim}

\section{Some Guidelines for Using Standard Facilities}

The following notes may help you achieve the best effects with the FAC
style file.

\subsection{Sections}

\LaTeX\ provides five levels of section headings and they are all
defined in the FAC style file:
%
\begin{itemize}
  \item \verb"\section"
  \item \verb"\subsection"
  \item \verb"\subsubsection"
  \item \verb"\paragraph"
  \item \verb"\subparagraph"
\end{itemize}
%
Section numbers are given for sections, subsection and subsubsection
headings.

\subsection{Running Headlines}

As described above, the title of the article and the author's name (or
authors' names) are used as running headlines at the top of every page.
The \verb"\pagestyle" and \verb"\thispagestyle" commands should {\em not\/}
be used. Similarly, the commands \verb"\markright" and \verb"\markboth"
should not be necessary.

\subsection{Illustrations (or Figures)}

The FAC style will cope with most positioning of your illustrations and you
should not normally use the optional positional qualifiers on the
\verb"figure" environment which would override these decisions.
See `Instructions for authors' in FAC journal for submission of artwork.
Figure captions should be below the figure itself therefore the
\verb"\caption"
command should appear after the figure or space left for an illustration.
For example, Figure~\ref{sample-figure} is produced using the following
commands:
%
\begin{verbatim}
  \begin{figure}
     \centering
     \vspace{5cm}
     \caption{An example figure in which space has been
              left for the artwork}
     \label{sample-figure}
  \end{figure}
\end{verbatim}
%
  \begin{figure}
     \centering
     \vspace{5cm}
     \caption{An example figure in which space has been
              left for the artwork}
     \label{sample-figure}
  \end{figure}

\subsection{Tables}
\begin{table}
  \caption{Index of Symbols}
  \begin{tabular}{ll}
   \hline
   Symbol\hspace{1cm}  & Meaning \\
   \hline
%  \whbl               & \hbox{Forward closure} \\
%  \blwh               & \hbox{Backward closure} \\
%  \blbl               & \hbox{Overlap closure} \\
%  \whwh               & \hbox{Rule closure} \\
   $\leadsto$          & \hbox{Rule closure constructor} \\
   $\bigtriangleup$    & \hbox{Substitution sum} \\
   $\cdot$             & \hbox{Substitution composition} \\
   $\sqcup$            & \hbox{Substitution join} \\
   $\bot$              & \hbox{Coherence relation} \\
   $-^n$               & \hbox{Substitution operator
                                         (exponent)} \\
   $W_n(-,-)$          & \hbox{Substitution operator
                                         (Whale)} \\
   $T_n(-,-,-)$        & \hbox{Substitution operator
                                         (Turtle)} \\
   \hline
  \end{tabular}
  \label{symbols}
\end{table}

The FAC style will cope with most positioning of your tables
and you should not normally use the optional positional qualifiers on the
\verb"table" environment which would override these decisions.
Table captions should be at the top, therefore the \verb"\caption" command
should appear before the body of the table.

The \verb"tabular" environment should be used to produce ruled tables;
it has been modified for the FAC style in the following ways:
%
\begin{enumerate}
  \item Rules may either be two-thirds or the full width of a page,
depending on the width of the material in your table. \verb"\hline"
will produce a rule two-thirds the width and \verb"\fullhline" will
produce a rule the full width of a page;
  \item Additional vertical space is inserted on either side of a rule;
  \item Vertical lines are not produced.
\end{enumerate}
%
Commands to redefine quantities such as \verb"\arraystretch" should be
omitted.
For example, Table~\ref{symbols} is produced using the following
commands:
%
\begin{verbatim}
  \begin{table}
    \caption{Index of Symbols}
    \begin{tabular}{ll}
     \hline
     Symbol\hspace{1cm}  & Meaning \\
     \hline
   %  \whbl          & \hbox{Forward closure} \\
   %  \blwh          & \hbox{Backward closure} \\
   %  \blbl          & \hbox{Overlap closure} \\
   %  \whwh          & \hbox{Rule closure} \\
   $\leadsto$        & \hbox{Rule closure constructor} \\
   $\bigtriangleup$  & \hbox{Substitution sum} \\
     ...
     $T_n(-,-,-)$    & \hbox{Substitution operator
                                          (Turtle)} \\
     \hline
    \end{tabular}
    \label{symbols}
  \end{table}
\end{verbatim}

\subsection{Displayed Mathematics}

The FAC style will set displayed mathematics with the correct indent
provided you use the \LaTeX\ standard of open and closed square brackets
as delimiters. The following equation
  \[ \sum_{i=1}^p \lambda_i = {\rm trace}({\bf S}) \]
was typeset in the FAC style using the commands:
\begin{verbatim}
  \[ \sum_{i=1}^p \lambda_i = {\rm trace}({\bf S}) \]
\end{verbatim}
Note the difference between the positioning of this equation and of
the following centred equation
  $$ \alpha_{j+1} > \bar{\alpha}+ks_{\alpha} $$
which was (wrongly) typeset using double dollars as follows:
\begin{verbatim}
  $$ \alpha_{j+1} > \bar{\alpha}+ks_{\alpha} $$
\end{verbatim}

\subsection{Bibliography}

References to published literature should be quoted in text by an
abbreviation in square brackets of name(s) (three letters) and date (two
digits). See examples below for style.
This is consistent with the Bib\TeX\ bibliography style `\verb"alpha".'
Where more than one reference is cited having the author(s) and date, the
letters a,b,c, \ldots\ should follow the date (\eg [Smi88a], [Smi88b],
\etc.). References should be listed in \verb"thebibliography" environment
alphabetically by author(s)' name(s) and then by year if the same author
has several papers.

The following listing shows some references prepared in the style of the
journal:
%
\begin{verbatim}
  \begin{thebibliography}{Lam86}
   \bibitem[But81]{But:Copy}
   Butcher, J.:
   {\em Copy-editing: The Cambridge Handbook.}
   Cambridge University Press, 1981.

   \bibitem[Chi69]{Chicago}
   {\em The Chicago Manual of Style.}
   University of Chicago Press, Chicago 60637, USA, 1982.

   \bibitem[For84]{For:Program}
   Forgaard, R.:
   A Program for Generating and Analyzing Term Rewriting
   Systems,
   Master's Thesis, MIT Lab. for Computer Science, 1984.

   \bibitem[JLR82]{Jou:Recursive}
   Jouannaud, J. P., Lescanne, P. and Reinig, F.:
   Recursive Decomposition Ordering,
   {\em Proc.\ Conf. on Formal Description of Programming
   Concepts II}, \,pp.\ 331--346, 1982.

   \bibitem[Lam86]{Lam:LaTeX}
   Lamport, L.:
   {\em LaTeX: A Document Preparation System.}
   Addison-Wesley, New York, 1986.

   \bibitem[Ped85]{Ped:Obtaining}
   Pederson, J.:
   Obtaining Complete Sets of Reproductions and Equations
   without Using Special Unification Algorithms.
   Unpublished manuscript, 1985.

   \bibitem[PeS81]{Pet:Complete}
   Peterson, G. E. and Stickel, M. E.:
   Complete Sets of Reductions for Some Equational
   Theories. {\em J.\ ACM}, {\bf 28}, 223--264 (1981).
  \end{thebibliography}
\end{verbatim}
%
This produces the following references:
%
\begin{thebibliography}{Lam86}
%
\bibitem[But81]{But:Copy}
Butcher, J.:
{\em Copy-editing: The Cambridge Handbook.}
Cambridge University Press, 1981.

\bibitem[Chi69]{Chicago}
{\em The Chicago Manual of Style.}
University of Chicago Press, Chicago 60637, USA, 1982.

\bibitem[For84]{For:Program}
Forgaard, R.:
A Program for Generating and Analyzing Term Rewriting Systems,
Master's Thesis, MIT Lab. for Computer Science, 1984.

\bibitem[JLR82]{Jou:Recursive}
Jouannaud, J. P., Lescanne, P. and Reinig, F.:
Recursive Decomposition Ordering,
{\em Proc.\ Conf. on Formal Description of Programming Concepts
II}, \,pp.\ 331--346, 1982.

\bibitem[Lam86]{Lam:LaTeX}
Lamport, L.:
{\em LaTeX: A Document Preparation System.}
Addison-Wesley, New York, 1986.

\bibitem[Ped85]{Ped:Obtaining}
Pederson, J.:
Obtaining Complete Sets of Reproductions and Equations without Using
Special Unification Algorithms. Unpublished manuscript, 1985.

\bibitem[PeS81]{Pet:Complete}
Peterson, G. E. and Stickel, M. E.:
Complete Sets of Reductions for Some Equational Theories. {\em J.\ ACM},
{\bf 28}, 223--264 (1981).
%
\end{thebibliography}

\label{lastpage}

\end{document}
