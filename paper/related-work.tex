
\section{Additional Related Work}
\label{sec:related-work}

Our proof was performed in 1996.  In the same year, 
Gonthier~\cite{gonthier1996verifying}
verified a detailed 
implementation of a realistic concurrent garbage collector~\cite{doligez1994portable} using TLP, a prover
for the Temporal Logic of Actions.  Gonthier's proof demonstrates 
that the implementation preserves a complex safety invariant 
with about 22,000 lines of proof.  Since 1996, there have been a
number of
verification efforts aimed at the verification of garbage collectors.
Jackson~\cite{jackson1998verifying} used an embedding of temporal logic in PVS to verify both
safety and liveness properties for an abstract
mutator/allocator/collector model of the tricolor algorithm of
Dijkstra, \emph{et al}.  This abstract model is then
refined to a lower-level heap-based implementation.  Burdy formalized
our refinement argument in both B and Coq for the purpose of comparing
the
two formal systems.  In Burdy's formalization, the abstract mutator
already colors the target of a pointer assignment.
Gammie, Hosking,
and Engelhardt~\cite{gammie2015relaxing} describe the Isabelle/HOL
formalization and verification of
the tricolor concurrent garbage 
collector (similar to the one verified by Gonthier) for an x86-TSO
memory model in a multi-mutator setting as an invariance proof.
Many of the proofs build in the cooperative marking by the mutator
into the specification.  When this marking is viewed as a refinement,
it
is important to demonstrate that the refinement has not restricted the
mutator so that it does not generate any garbage.  It can do this, for
example,
by never redirecting a pointer so that a node is orphaned.  Such a
mutator
would satisfy the refinement with an idle garbage collector.  A
correct
refinement must preserve the nondeterminism of the client and must
therefore must simultaneously witness a simulation relation on the
collector and a bisimulation
relation on the mutator.  

Several efforts cover non-concurrent garbage collectors.  McCreight,
Shao, Lin, and Li~\cite{mccreight2007general} use Coq to
verify the safety of the implementation of several
stop-the-world and incremental garbage collectors in an assembly
language.  Coupet-Grimal and Nouvet~\cite{8133460} embed temporal
logic in Coq to verify an incremental garbage collection algorithm.
Hawblitzel and Petrank~\cite{hawblitzel2009automated} verify stop-the-world garbage collectors using
Boogie exploiting the quantifier instantiation capability of the Z3
SMT solver. Ericsson, Myreen, and Pohjola~\cite{ericsson2017verified}
describe the verification of
the CakeML generational garbage collector in HOL4.


%LEFT OUT DUE TO SPACE (BUT IT IS GOOD INFO)
%
%The garbage collector  has  also been  specified  and 
%verified in  the
%UNITY  framework~\cite{CM:UNITY} using a  notion of  
%refinement called
%{\em superposition}.  This refinement notion differs  from 
%ours in the
%sense that the initial algorithm (specification)  is not 
%regarded as a
%specification of lower levels.   Rather   it is  taken as 
%an   initial
%starting point that is then enriched with more details 
%until the final
%enrichment   represents the full   algorithm.  Each level 
%inherits the
%properties proved about the previous  level, such that the 
%final level
%inherits  all the properties proven  for all previous 
%levels, and this
%combined collection of  properties can then be used  to 
%prove that the
%final level satisfies  the desired  safety  and liveness  
%properties.  
%Only the marking phase is verified in this exercise.