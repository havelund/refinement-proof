         
\section{Introduction}

Russinoff~\cite{Rus:GC} used   the Boyer-Moore theorem prover  to
verify  a  safety  property   of   a {\em  mark--and--sweep}   garbage
collection  algorithm originally suggested  by Ben-Ari \cite{Ben:GC}. 
The garbage collector and   its composition with   a user  program  is
regarded as a concurrent system with both processes working on
a common shared memory.  The  collector uses a colouring  (marking) technique
to iteratively colour all accessible nodes {\em black} while 
leaving garbage nodes {\em white}.  When the colouring has
stabilized, all the white nodes can be collected and placed in the
free list.

Russinoff's correctness property is formulated as a  state predicate $P$, which
is then proven to be an invariant, i.e., true in all reachable  states.
In gross  terms, 
this invariant predicate is formulated as  follows.  The garbage collector
can at 
any time be in one of 9 different locations.  In one of the locations,
here  called {\sc  Append},  the {\em append} operation  representing
garbage collection is applied to a certain  memory node $X$, but only
when this node is white. The safety  predicate  $P$  is  then formulated  as:

\begin{center}
  If the control of the garbage collector is at\\ location {\sc Append} and
  $X$ is white then $X$ is garbage 
\end{center}

\noindent
This formulation of   the  safety property does not really  tell us
whether the  program is correct.  We 
have to additionally ensure that  the {\em append} operation is
only invoked in location {\sc Append}, and only on white nodes.  Hence,
the safety property of the garbage collector follows from both the
invariance of $P$ and an operational understanding of the
garbage collection algorithm.  

This  observation motivated  us  to  carry  out a   proof  in PVS
\cite{Owre95:prolegomena} using   a  refinement approach,  where   the
safety property  itself  is formulated  as an  abstract algorithm, and
the proof  is  based  on refinement  mappings  as suggested  by
Lamport~\cite{TLA:TOPLAS94}.  This  approach   has the advantage  that  the
safety property can  be formulated more abstractly without considering
the internal structure of the final implementation.  Here a {\em black
  box} view   of the  algorithm is   sufficient.  This yields a
further contribution in terms of the 
formalization of refinement mappings in PVS.
In  order   better to make  a
comparison, we also carried out a proof in PVS using the same
technique   as  in~\cite{Rus:GC}.  This    work   was documented  in
\cite{havelund-pvs-gc-99}.

In~\cite{HS:BRP}  we verified  a  distributed communication  protocol
using similar techniques for representing  state transition systems. 
The work  presented here can therefore be   seen as part of a
series of  PVS verifications of parallel/distributed algorithms, 
carried out to establish a framework for such verifications, while
identifying the main difficulties  in carrying  out such
proofs. Our  key conclusion  is  that  techniques for
strengthening invariants are  of major importance in refinement
proofs as well as in proofs of temporal properties.  
%
The proof presented in this paper was carried out over two decades ago, but
was never published. Since we still consider the work relevant, and even
cited, we decided to finally publish this work.

An initial  version of the algorithm  was  first proposed by Dijkstra,
Lamport, Martin, Scholten, and Steffers~\cite{DLMSS:GC} as an exercise
in  organizing and verifying the cooperation  of concurrent processes. 
Their solution   involved   three  colours.   Ben-Ari    improved this
algorithm so  as  to use    only  two colours  while  simplifying  the
resulting proof.  All of these  proofs were  informal {\em pencil  and
  paper\/} exercises.  As   pointed  out by   Russinoff~\cite{Rus:GC},
these informal proofs  ran into difficulties  of one sort or  another. 
Dijkstra, et al~\cite{DLMSS:GC}    explained   (as an  example  of   a
``logical trap'') how they originally proposed a minor modification to
the  algorithm.   This claim turned   out  to   be  wrong, and  was
discovered by the authors just  before the proof reached publication.  
Ben-Ari  later  proposed the same  modification   to his algorithm and
argued for   its    correctness   without  discovering its       flaw. 
Counterexamples    were  later  given    by  Pixley~\cite{Pix:GC}  and
van~de~Snepscheut~\cite{Van:GC}.   Furthermore,  although    Ben-Ari's
algorithm (which is the one we verify in PVS) is correct, his proof of
the safety property was found to be flawed.  This flaw was essentially
reproduced by Pixley\cite{Pix:GC} where it  again survived the  review
process, and was only discovered  ten years later by Russinoff  during
the course of his mechanical verification~\cite{Rus:GC}.  Ben-Ari also
gave a flawed  proof of a liveness  property ({\em every garbage  node
  will eventually be collected}) that was later observed and corrected
by van~de~Snepscheut~\cite{Van:GC}.

The garbage collector  has  also been  specified  and verified in  the
UNITY  framework~\cite{CM:UNITY} using a  notion of  refinement called
{\em superposition}.  This refinement notion differs  from ours in the
sense that the initial algorithm (specification)  is not regarded as a
specification of lower levels.   Rather   it is  taken as an   initial
starting point that is then enriched with more details until the final
enrichment   represents the full   algorithm.  Each level inherits the
properties proved about the previous  level, such that the final level
inherits  all the properties proven  for all previous levels, and this
combined collection of  properties can then be used  to prove that the
final level satisfies  the desired  safety  and liveness  properties.  
Only the marking phase is verified in this exercise.

In  Section~\ref{transition-systems},  a    formalization of state
transition  systems and  refinement mappings  is  provided in  an informal
mathematical style that  is  later formalized  in  PVS.  The garbage
collection algorithm is described in
Section~\ref{informal-specification}\@.
Sections~\ref{goto-specification} and~\ref{refinement-steps} present
the successive refinements of the initial algorithm in three stages.
This presentation is based on   an informal notation for  transition systems.
Section~\ref{observations} lists some observations on the entire
verification exercise. 
Appendices \ref{pvs-specifications} and \ref{pvs-proof} formalize
the concepts  introduced   in   Sections~\ref{transition-systems},
\ref{goto-specification} and~\ref{refinement-steps} in PVS.   

%%% Local Variables: 
%%% mode: latex
%%% TeX-master: t
%%% End: 


