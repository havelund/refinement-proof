
\newcommand{\bm}{BM$_{inv}$}
\newcommand{\pvsinv}{PVS$_{inv}$}
\newcommand{\pvsref}{PVS$_{ref}$}
\newcommand{\unity}{UNITY$_{ref}$}


\section{Observations}
\label{observations}

It is possible  to compare the  present proof (\pvsref-proof) with two
other  mechanized proofs of exactly  the same algorithm:  the proof in
the Boyer-Moore prover \cite{Rus:GC},  from now on  referred to as the
\bm-proof; and   the  PVS proof   \cite{havelund-pvs-gc-99},  
referred  to as  the \pvsinv-proof.  Instead of being based on refinement, these two proofs
are  based on a statement of  the correctness criteria as an invariant
to be  proven about the  implementation (the third refinement step)\@. 
The \pvsinv-proof follows  the \bm-proof closely.  Basically  the same
invariants were needed. The \pvsref-proof has the advantage over the two
other proofs, that the correctness criteria can be appreciated without
knowing the internal structure of  the implementation.  That is, we do
not need to know for example that the append operation is only applied
in location  {\sc  Append} to node  $X$,  and only  if  $X$ is white.  
Hence, from  this  perspective,  the  refinement proof  represents  an
improvement.  The \pvsref-proof has approximately the same size as the
\pvsinv-proof, in that basically  the same invariants and lemmas about
auxiliary functions  need to  be proven (19   invariant lemmas and  57
function lemmas). The  proof effort took a couple   of months.  Hence,
one  cannot argue that  the proof   has  become any  simpler.   On the
contrary in fact: since  we have many levels, there  is more to prove. 
Some   invariants were easier   to   discover when using   refinement,
especially  at the  top  levels. In  particular  nested  loops may  be
treated  nicely with refinement, only introducing  one loop at a time. 
In general, loops in  the algorithm to be  verified are the reason why
invariant discovery is hard, and of course nested loops are no better.
The main lesson  obtained from the \pvsinv-proof  is the importance of
invariant discovery in   safety   proofs.  Our experience  with    the
\pvsref-proof is  that refinement does not  relieve us of  the need to
search for   invariants. We  had to  come   up with exactly   the same
invariants in both cases, but the discovery process was different, and
perhaps more structured in the refinement proof.
Automated or semi-automated discovery of invariants remains a challenging research topic.

%For future  work one can  consider investigating 
%ways of automatically
%generating  invariants.  Since this is very  hard 
%in  general, one can
%alternatively consider more pragmatics ways of 
%supporting the PVS user
%in identifying new invariant  lemmas. For 
%example,  a tool that  helps
%the  user to infer new  invariants from unproven 
%sequents generated by
%the PVS system as a result of failed proofs.


%%% Local Variables: 
%%% mode: latex
%%% TeX-master: t
%%% End: 
